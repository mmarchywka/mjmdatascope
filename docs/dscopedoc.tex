
%%%%%%%%%%%%%%%%%%%%%%%%%%%%%%%%%%%%%%%%%%%%%%%%%%%%%%%%%%%%%%%%%%
% Sample template for MIT Junior Lab Student Written Summaries
% Available from http://web.mit.edu/8.13/www/Samplepaper/sample-paper.tex
% Last Updated April 12, 2007
% Adapted from the American Physical Societies REVTeK-4 Pages
% at http://publish.aps.org

\setlength{\paperheight}{11in}
% http://tex.stackexchange.com/questions/74636/mla-package-and-thumbpdf
\makeatletter
\@namedef{ver@thumbpdf.sty}{}
\makeatother
\documentclass[aps,secnumarabic,balancelastpage,amsmath,amssymb,nofootinbib]{revtex4}

\input{myskeletonpackages.tex}

\newcolumntype{.}[1]{D{.}{.}{#1}}
%\usepackage[maxfloats=30]{morefloats}   %mjm   saving up figs for the end   
% no param on old version stuck at 36
\usepackage{morefloats}        %mjm   saving up figs for the end   
\usepackage{graphicx}        %mjm   saving up figs for the end   

% will need modificaitons 
\input{recent_template.tex}



% https://tex.stackexchange.com/questions/121601/automatically-wrap-the-text-in-verbatim
\usepackage{listings}
\lstset{
basicstyle=\small\ttfamily,
columns=flexible,
breaklines=true
}

% none of this fking fking works for a f F 
\newcommand{\mjmverbatim}{lstlisting}
\newcommand{\mjmbeginverbatim}{\begin{lstlisting}}
\newcommand{\mjmendverbatim}{\end{lstlisting}}

\newcommand{\mjmmangle}[1]{keep/#1}



%# CHANGE VERSION AND STATUS MANUALLY 
% need a draft/notes/release flag

% https://tex.stackexchange.com/questions/5894/latex-conditional-expression
%At the command-line, you can do \def\MYFLAG{} and then test if \MYFLAG is defined in your document (or an included style file) with \ifdefined\MYFLAG ... \else ... \fi.
% needs trailing space for the sample bibtex doh
% leading spaces mess up the entry thought 
\def\xxmjmrelease{0.10 }
\ifdefined\mjmrelease
\newcommand{\mjmstatus}{ PUBLIC NOTES }
\newcommand{\mjmversion}{\mjmrelease} %%%%%%%%%%%%%
\newcommand{\mjmtrno}{MJM-2023-007}
\newcommand{\mjmbib}{\mjmtrno-\mjmversion}
\newcommand{\mjmstatuswarn}{{\bf{  }}   }
% 2021-09-29 wanted version wth for brownie
%\newcommand{\mjmbib}{\mjmtrno-\mjmversion-\mjmrelease}
%\newcommand{\mjmbib}{\mjmtrno}
\else
\newcommand{\mjmstatus}{ NOT public NOTES }
\newcommand{\mjmversion}{0.00} %%%%%%%%%%%%%
\newcommand{\mjmtrno}{MJM-2023-007}
%\newcommand{\mjmbib}{\mjmtrno-\mjmversion}
\newcommand{\mjmbib}{\mjmtrno}
%\newcommand{\mjmstatuswarn}{  }
\newcommand{\mjmstatuswarn}{{\bf{This document is a non-public DRAFT and contents may be speculative or undocumented or simple musings and should be read as such.  }}   }
\fi

%\newcommand{\mjmstatus}{ NOT public NOTES }



\newcommand{\mjmtitle}{MJMDatascope : A viewer for streamed numerical data}
\begin{comment}
\newcommand{\mjmmakedate}{2023-09-14 }
\newcommand{\mjmauthor}{Mike J Marchywka }
\newcommand{\mjmbasename}{\jobname}
\newcommand{\mjmaddbio}{mjm_tr,releases}
\newcommand{\mjmversion}{ 0.00 }
\newcommand{\mjmtrno}{MJM-2023-007}
\newcommand{\mjmbibday}{14}
\newcommand{\mjmbibmo}{09}
\newcommand{\mjmbibyear}{2023}
\newcommand{\mjmemail}{marchywka@hotmail.com}
\end{comment}

\input{mycommands.tex}


\newcommand{\mjmauthor}{Mike J Marchywka }
\newcommand{\mjmmakedate}{2023-09-14 }
\newcommand{\mjmbasename}{\jobname}
\newcommand{\mjmaddbio}{mjm_tr,releases}
%\newcommand{\mjmversion}{0.00}
%\newcommand{\mjmtrno}{MJM-2023-007}
%\newcommand{\mjmbibday}{14}
%\newcommand{\mjmbibmo}{09}
%\newcommand{\mjmbibyear}{2023}

% the build script changes these to creation day doh 
\newcommand{\mjmbibday}{14}
\newcommand{\mjmbibmo}{09}
\newcommand{\mjmbibyear}{2023}


\newcommand{\mjmmakebibday}{\number\day}
\newcommand{\mjmmakebibmo}{\number\month}
\newcommand{\mjmmakebibyear}{\number\year}

\newcommand{\mjmbibtype}{techreport}

\newcommand{\mjmbibname}{marchywka-\mjmbib}
\mjmstartbib{\mjmbibtype}{\mjmbibname}

\newcommand{\mjmprod}{MJMDatascope}


\newcommand{\mjmabstract}{
\mjmprod is designed to be a lightweight viewer or previewer for
streaming numerical data from a variety of sources such as
"R", FEM codes such as FreeFem++, and custom or proprietary
numerical code with a current interface for c++.
It is not designed to be particularly feature rich or
generate publication quality output but mostly to monitor
and compare number streams from disparite sources while
providing a uniform user interface.
The current form ( 2023-09-14)  is proof of concept and not usable
by most users except developers with an interest in numerical
codes. Note that with a simple interface, it could be
used to plot streams of numbers from stdin making it
useful for a variety of monitoring tasks.
}



\newcommand{\mjmemail}{marchywka@hotmail.com}
%\newcommand{\mjmaddr}{306 Charles Cox , Canton GA 30115}
\newcommand{\mjmaddr}{44 Crosscreek Trail, Jasper GA 30143}
\mjmaddbib{title}{\mjmtitle}
\mjmaddbib{author}{\mjmauthor}
\mjmaddbib{type}{\mjmbibtype}
\mjmaddbib{abstract}{\mjmabstract}
%\mjmaddbib{name}{marchywka-\mjmbib}
\mjmaddbib{name}{\mjmbibname}
\mjmaddbib{number}{\mjmtrno}
\mjmaddbib{version}{\mjmversion}
\mjmaddbib{institution}{not institutionalized, independent }
\mjmaddbib{address}{ \mjmaddr}
\mjmaddbib{date}{\today}
\mjmaddbib{startdate}{\mjmbibyear -\mjmbibmo -\mjmbibday }
%\mjmaddbib{day}{\mjmbibday}
%\mjmaddbib{month}{\mjmbibmo}
%\mjmaddbib{year}{\mjmbibyear}
\mjmaddbib{day}{\mjmmakebibday}
\mjmaddbib{month}{\mjmmakebibmo}
\mjmaddbib{year}{\mjmmakebibyear}

\mjmaddbib{author1email}{\mjmemail}
\mjmaddbib{contact}{\mjmemail}
\mjmaddbib{author1id}{orcid.org/0000-0001-9237-455X}
\CatchFileEdef\mjmpages{\mjmbasename.last_page}{\endlinechar=-1\relax}
% TODO FIXME add this to the skeleton text 
%\mjmaddbib{pages}{ \input{\mjmbasename.last_page}}
\mjmaddbib{pages}{ \mjmpages}
%\mjmaddbib{filename}{\mjmbasename}

\begin{comment}
\newcommand{\checkfortwo}[1]{%
  \ifcsname#1\endcsname%
  VERSION =        \{\mjmversion \today \mjmstatus \},
  \else%
  VERSION =        \{\mjmversion\},
  \fi%
}



\newcommand{\xxxmjmfullbib}{
{@}TECHREPORT\{mmarchywka-\mjmbib,
  AUTHOR =        \{M.J. Marchywka\},
  TITLE =         \{ \mjmtitle \},
  NUMBER =        \{\mjmtrno\},
\checkfortwo{mjmstatus}
  %VERSION =        \{\mjmversion \today \mjmstatus \},
  %VERSION =        \{\mjmversion\},
 INSTITUTION =   \{  not institutionalized , independent\},
  ADDRESS =       \{\mjmaddr\},
  NOTE =          \{Version \mjmversion , may change signficantly if less than 1.00   \},
  DATE  =          \{\today\},
  DAY  =          \{\mjmbibday\},
  MONTH =         \{\mjmbibmo\},
  YEAR  =         \{\mjmbibyear\},
 % AUTHOR1EMAIL = \{marchywka@hotmail.com\},
  AUTHOR1EMAIL = \{\mjmemail\},
  AUTHOR1ID = \{orcid.org/0000-0001-9237-455X\},
%   PAGES =         \{6\},
  %PAGES =         \{ \input{\mjmbasename.last_page} \},
  PAGES =         \{ \mjmpages \},
  CONTACT =       \{marchywka@hotmail.com\},
  FILENAME =       \{\mjmbasename\}
\}
} %mjmfullbib

\end{comment}
%\mjmaddbib{bibtex}{\mjmfullbib}

\mjmdonebib



\lhead{\mjmauthor,  \mjmtrno }


%\lhead{M Marchywka,  \mjmtrno }
%\rhead{ \mjmversion not for public release}
%\rhead{ { \today }  v. \mjmversion for release without review }
%\rhead{ { \today }  v. \mjmversion NOT public DRAFT }
%\rhead{ { \today }  v. \mjmversion { }  NOT public NOTES }
\rhead{ { \today }  v. \mjmversion { }  \mjmstatus }

\newfloatcommand{capbtabbox}{table}[][\FBwidth]


%%%% build flags 
%\newlength{\desttabw}  \setlength{\desttabw}{4in}
\newlength{\desttabw}  \setlength{\desttabw}{\textwidth}
\newlength{\chainwidth}  \setlength{\chainwidth}{.4\textwidth }
\newlength{\slantwidth}  \setlength{\slantwidth}{.2\textwidth }
\newlength{\subfigwidth}  \setlength{\subfigwidth}{.3\textwidth }
\newlength{\fullfigwidth}  \setlength{\fullfigwidth}{.8\textwidth }
\newlength{\subwfigwidth}  \setlength{\subwfigwidth}{.75\textwidth }
\newlength{\subwfigwidthrot}  \setlength{\subwfigwidthrot}{\textwidth }
\newlength{\myboxwidth}  \setlength{\myboxwidth}{.3\textwidth }
\newlength{\picwidth}  \setlength{\picwidth}{.4\textwidth }
% set to center for nowmal output 
\newcommand{\destflushtab}{flushleft}
\includecomment{mdpicomment}
\excludecomment{draftcomment}
\excludecomment{badmathcomment}
\excludecomment{showworkcomment}
% this does not fing work ... 

\newcommand{\mjmed}[1]{
%\begin{mjmedx} 
[ mjm : #1   ]
%\end{mjmedx} 
}  

% thinking outload
\newcommand{\mjmtolx}[1]{}
\newcommand{\mjmtolxx}[1]{}
\newcommand{\mjmtol}[1]{
 \fbox{  
% thi does not ing work right 
\begin{minipage}[t]{\textwidth}
{ \centering{\bf{Thinking outloud}} }
\par   
#1 
\end{minipage} 
}
}
%\mjmpicture{Rscope.jpeg}{Output to \mjmprod using the R example}{Rdscope}

\newcommand{\mjmpicture}[3]
{
\begin{figure}[H]
{ \includegraphics[height=3in,width=4in]{keep/#1} }
\caption{#2}
\label{fig:#3}
\end{figure}
} % mjmpicture


\newcommand{\mjmaside}[1]{
 \fbox{  
% thi does not ing work right 
\begin{minipage}[t]{\textwidth}
{ \centering{\bf{Aside: }} }
\par   
#1 
\end{minipage} 
}
}




\newcommand{\mjmgraphics}[1]{#1 }
\newcommand{\mjmfullplot}[1]{\includegraphics[width=\fullfigwidth]{#1}}
%\newcommand{\mjmincludeplot}[1]{\includegraphics[width=\fullfigwidth]{#1}}
%\newcommand{\mjmincludeplot}[1]{\includegraphics[width=\subfigwidth]{#1}}
\newcommand{\mjmincludeplot}[1]{\includegraphics[height=3in,width=\fullfigwidth]{#1}}

% include here as likely to be doc specific 
%\newcommand{\mjmreffig}[1]{Fig. \ref{#1}}
\newcommand{\mjmreffig}[1]{Fig. \ref{fig:#1}}
\newcommand{\mjmreftab}[1]{Table  \ref{tab:#1}}


%cp yyy2.pdf ~/d/latex/keep/pp20171124biotin.pdf
\newcommand{\mjmdatedplot}[1] 
{ \includegraphics[height=3in,width=\fullfigwidth]{keep/pp20171124#1} }
% right now there are too many figs to save for the end apaprently
% hard limit is 36
\newcommand{\mjmbeginfigure}{\begin{figure}[H] }
%\newcommand{\mjmbeginfigure}{\begin{figure}[p] }
\newcommand{\mjmfigure}[1]{
%\begin{figure}[H]
\mjmbeginfigure

#1 

\end{figure}
}
%\extrafloats{100}

\newcommand{\mjmlisting}[1]
{
\begin{lstlisting} 
#1 
\end{lstlisting}
}

\newcommand{\mjmold}[1]{ } 



\newcommand{\mjmeqn}[1]{\begin{equation} #1 \end{equation}  } 





\begin{document}

\title{\mjmtitle}
\author         {Mike Marchywka}
\email          {\mjmemail}
\thanks{ to cite  or credit this work, see bibtex in Appendix \ref{appendix:citing} } 
\date{\today}
\affiliation{\mjmaddr}

\mjmblackboxno{Release Notes  xxxx-xx-xx : }{
not ready at all


\mjmreleasewarning
\mjmstatuswarn

\mjmexplainbib

\mjmreleasewarning

\mjmwarnfeed

\mjmwarnme

\mjmwarntopic

\bf{ Readers may experience discalimer fatigue. Doe not proceed if you 
are weary or unable to thnk clearly}


}

\begin{abstract}
\mjmabstract
\begin{comment}
\mjmprod is designed to be a lightweight viewer or previewer for
streaming numerical data from a variety of sources such as
"R", FEM codes such as FreeFem++, and custom or proprietary
numerical code with a current interface for c++. 
It is not designed to be particularly feature rich or
generate publication quality output but mostly to monitor
and compare number streams from disparite sources while
providing a uniform user interface.
The current form ( 2023-09-14)  is proof of concept and not usable
by most users except developers with an interest in numerical
codes. Note that with a simple interface, it could be
used to plot streams of numbers from stdin making it
useful for a variety of monitoring tasks.
\end{comment}
\end{abstract}

\maketitle
\tableofcontents
\newpage


\section{Introduction  }

Many numerical codes exist in a spectrum of applications from
comprehensive symolic math packages such as Mathematica ,
distributed simulators running on supercomputers, personal
FEM codes, to just generation of  a list of numbers piped
in from stdin. In most cases, results are sent to a file
for analysis off line or plotted synchronously at
various times. Display applications vary widely too.
As the data types are diverse, it may be expceted that diverse
viewers also exist and are optimized for each type. However,
for a range of applications a simple unobtrusive 
monitoring system may be useful and that is the niche 
targeted by \mjmprod.  

Numerical data generating code interfaces to the viewer
only via a connection such as a linux fifo or 
UDP packets cleanly separating the generator from the 
unreliable consumer. Currently, interfaces are provided
for Freefem++,R, and most c++ code although the range
of sources should be easily expanded. 

All aspects of \mjmprod are currently being improved
including the API. The current graphics implementation
is largely in GLUT, as was my code from 2007. There is some
possibility of doing this with a browser plugin but
that removes the "lightweight" attribute. 

\section{History and Context  }
Going back at least to 2007, I wanted to be able to stream simple
data in real time into a display application and began writing
some simple code but quickly dropped the probject. While
learning to use Freefem++, I noted that their graphics 
was organized in a similar way to my earlier interests but
there was no obvious easy was to plot 1D data. In any case,
I needed that to develop my own finite difference code
and make comparisons easy.

In a quick discussion on the R devel mailing list, the
package Trend was brought to my attention and this is
very similar to my earlier interests. It is possible that
that pacakge or an adaptation of the Freefem++ graphics
code may be made to work similarly to \mjmprod.

\section{ Download, Build, Install  }

Currently the proof of concept code has been put up
on github under the mmarchywka account.
The components include the viewer source code, configuration
files, and interfaces with examples and their configuration
files. All of the source code is written as c++ header
files making it easy to use a given class in an app
or as a standalone simple/test app. Building is just a matter
of one compile and link step. Most of this code goes into 
a header library on my local machine but I just merged
all the dependencies into this repo for now.  

\subsection{ \mjmprod }

\begin{lstlisting}
./run_datascope -compile
g++ -std=gnu++11 -DTEST_datascope__ -Wno-deprecated-declarations -gdwarf-3 -O0 -MMD -MF datascope.deps -I. -I/home/documents/cpp/mjm/hlib/ -I/home/documents/cpp/mjm/num -I/home/documents/cpp/mjm/ -I/home/documents/cpp/mjm/tcl -I/home/documents/cpp/mjm/include -I/home/documents/cpp/pkg/include -I/home/documents/cpp/pkg -I../freefem -I/usr/include/GL -Wall -Wno-unused-variable -Wno-unused-function -Wno-sign-compare -Wno-non-template-friend -Wno-misleading-indentation -x c++ ./datascope.h -o datascope.out -lreadline -lpthread -lGL -lglut -lGLU -lglut -lpng -lavcodec -lswscale -lavutil
In file included from ./mjm_glut_scope_ii.h:7,
                 from ./datascope.h:14:
./mjm_glut_saver.h:95:2: warning: #warning ate out deprecated crap [-Wcpp]
   95 | #warning ate out deprecated crap
      |  ^~~~~~~


\end{lstlisting}



\begin{lstlisting}
./datascope.out -source listenraw.txt  2>&1 
\end{lstlisting}


\subsection{ R interface  }


\begin{lstlisting}

 2340  R_util -buildins mjmdscope


build_ins()
{
pnm="$1"
pdir=`pwd`
#cat << ---xxx---
export PKG_CXXFLAGS=" -MMD -MF $pdir/$pnm.deps -I/home/documents/cpp/proj/datascope -I/home/documents/cpp/mjm/hlib -I/home/documents/cpp/proj/freefem -I/home/documents/cpp/mjm/num"
export PKG_LIBS="-lpthread -lreadline "
R CMD build "$pnm"
R CMD INSTALL ${pnm}_1.0.tar.gz
#---xxx---

\end{lstlisting}

\subsection{ Freefem++ interface }

\begin{lstlisting}


freefem_util -module  mjm_dscope_freefem



# make new module 
# https://doc.freefem.org/documentation/developers.html#a-first-example-myfunction-cpp
make_ff_module()
{
fn="$1"
#ff-c++ myfunction.cpp
I=" -lpthread -lreadline -I. -I../datascope -I../../mjm/hlib -I../../mjm/num -I/home/ubuntu/dev/freefem/FreeFem-sources-4.12/src/fflib"
# get deps
echo making deps file first 
g++ -fPIC  -c -g -MMD -MF $1.deps $I  -I/home/ubuntu/dev/freefem/install/ff-petsc/r/include -I/home/ubuntu/dev/freefem/install/lib/ff++/4.12/include/ $I  "$fn".cpp 
#g++ -bundle -undefined dynamic_lookup -g "$fn".o -o ./"$fn".dylib

if [ "$SKIP_MODULE" == "" ]
then
ff-c++ $I  "$fn".cpp
fi

#g++ -bundle  -undefined dynamic_lookup  -g "$fn".o -o ./"$fn".dylib
} # make_ff_module



\end{lstlisting}


\subsection{ A simple c++ app }

This is not a minimal example but the results should compare
to the Freefem++ example to illustrate the (vaporware) ability
of \mjmprod to compare similar data from various streams.

The compile command on my local machine is echoed by the script
and can be modified to suit any local install,  

\begin{lstlisting}
!2298
. quickg++ -last
using mjm_simple_1d_dd.h
g++ -Wall -Wno-misleading-indentation -MMD -MF mjm_simple_1d_dd.deps -std=gnu++11 -DTEST_MJM_SIMPLE_1D_DD -I. -I../datascope -I../../mjm/hlib -I../../mjm/num -gdwarf-3 -O0 -x c++ mjm_simple_1d_dd.h -o mjm_simple_1d_dd.out -lpthread -lreadline
mjm_simple_1d_dd.h: In instantiation of ‘void mjm_simple_1d_dd<Tr>::Run(const StrTy&, const StrTy&) [with Tr = Tr; mjm_simple_1d_dd<Tr>::StrTy = std::__cxx11::basic_string<char>]’:
mjm_simple_1d_dd.h:794:44:   required from ‘void mjm_simple_1d_dd<Tr>::run(const StrTy&, const StrTy&) [with Tr = Tr; mjm_simple_1d_dd<Tr>::StrTy = std::__cxx11::basic_string<char>]’
mjm_simple_1d_dd.h:1326:37:   required from here
mjm_simple_1d_dd.h:891:13: warning: unused variable ‘cc’ [-Wunused-variable]
  891 | const IdxTy cc=ComputeDn(dn,rv.dt,0);

\end{lstlisting}

\section{ Running the examples  }

At this point, there should be a \mjmprod executable
with a sample configuration file to begin listening on 
a default fifo. This is invoked with the command, 

\begin{lstlisting}
./datascope.out -source listenraw.txt  2>&1 


cat listenraw.txt 
# start the display although not needed right wawy
scope launch
# data client with no parameters
##dgram launch "ipv4s=192.168.4.1" 2 
# server 
#dgram launch "ipv4s=127.0.0.1" 3 
rawfifo launch "ipv4s=127.0.0.1" 3 
# client or data source a source is a source of course of course
#dgram launch "ipv4s=127.0.0.1" 2 
# heart beat send no params 
#send  "chunks=sinewave"
#scope saver
#scope spr width=600;height=600;mpeg;mpeg_name=foodoo.mpeg;fps=30 
#scope saver
displayfifo

\end{lstlisting}

At this point, a blank display window should appear along
with a command prompt("mjm>") and a lot of informational
messages. Failure to start could be due to inability to link
and running ldd on the executable may be a good starting point. 
lsof should show the fifo is opened. 

Obviously this needs to run to display the output of the below
examples but all should run regardless of the state
of \mjmprod. Ideally they will stop packaging data
for export if the send que fills up but this inot implemented
yet. Control facilities exist in the case of UDP that could stop
sending absent a negoatiated listener but that too is not implemeted. 


\subsection{ R }

Probably the example of most widespread interst would
be the R data source illustrated here. If the module was
built and installed the following code should
produce moving sine waves on the \mjmprod graphics screen.
The current version has the copious informational messages
suppressed but that can be restroed by rebuilding
the package with the global muting variable flipped. 


\mjmpicture{Rdscope.jpeg}{Output to \mjmprod using the R example}{Rdscope}
%ls keep | more ffdscope.jpeg Rdscope.jpeg simdscope.jpeg


\begin{lstlisting}
cat dscope.R 

#Rcpp::compileAttributes("mjmdscope")
library("mjmdscope")
nr=100.0
nc=5
df<- as.data.frame(matrix(ncol=nc,nrow=nr,data=0.0))
#df<-data.frame(nrows=nr,cols=nc)
for(k in 1:10000)
{
for(i in 1.0:nr)
{

for(j in 1.0:(nc-1))
{
df[i,j]=sin(.01*2.0*pi*j*i/nc+.0333*k*2*pi);
}  # j
df[i,nc]=1.0*(i)/nr;
} # i 
mjmdscope::mjmdscopesend(df)
Sys.sleep(.2)
} # k 
str(df)
\end{lstlisting}




\subsection{ Freefem++ }

This should demonstrate the simple time evolution of
a diffusion problem with a central source region
and a concentration dependent diffusivity. 
The problem is documented in ctryptic form in 
the configuration file for the following c++
code which solves a similar problem using finite difference. 

\mjmpicture{ffdscope.jpeg}{Output to \mjmprod using the Freefem++  example}{ffdscope}

\begin{lstlisting}
 2360  FreeFem++ -ne ddglue1d.edp  -dt .1 -kdiff 20
\end{lstlisting}


\subsection{ A c++ app }
The configuration file contains a terse description
of a diffusion probablm nominally identical to the 
above Freefem++ code. 

\mjmpicture{simdscope.jpeg}{Output to \mjmprod using the simple c++ fd  example}{cppdscope}
\begin{lstlisting}


2330  ./mjm_simple_1d_dd.out "source sime1d.txt"  quit 2>&1

cat sime1d.txt 
# correct for factor of 2/d_sigma or avg pt distance ( 1000 lol) 
add_point n=1000;d=.001
check
#kvp.get(m_nz,"nz"); kvp.get(m_ii,"nzed"); kvp.get(m_if,"nend");
#kvp.get(m_dz,"dz"); kvp.get(m_q,"q"); kvp.get(m_mu,"mu");
add_species TF nz=0;dz=0;q=0;mu=10;br0=.29999;bv0=0;br1=.3;bv1=1;br2=.6;bv2=1
add_species f1 nz=1;dz=.0000100;q=0;mu=0
add_species f1a nz=0;dz=.00005;q=0;mu=0
add_species f2 nz=1;dz=.00005;q=0;mu=0
add_species f2a nz=0;dz=.000001;q=0;mu=0
#add_species f3 nz=1;dz=.00005;q=0;mu=0
#add_species f3a nz=0;dz=.00001;q=0;mu=0
#add_species f4 nz=1;dz=.05;q=0;mu=0
#add_species f4a nz=0;dz=.01;q=0;mu=0
add_species fgn nz=1;dz=.0001;q=0;mu=0
add_species fb nz=0;dz=.0000001;q=0;mu=0
add_reaction f1a f1 ; TF , 1.0e-1 -> f1a
add_reaction f2a f2 ; f1a , 1.0e-1 -> f2a
#add_reaction f3a f3 ; f2a , 1.0e-1 -> f3a
#add_reaction f4a f4 ; f3a , 1.0e-1 -> f4a
#add_reaction fb fgn ; f1a , 1.0e-0 -> fb
add_reaction fb fgn ; f2a , 1.0e-0 -> fb
dscope "rawfifo launch  2"
list 
setup
# run n=0;dt=.01;sleep=0;itermod=100;geldof=fb;ef=20
run n=0;dt=.01;sleep=0;itermod=100;geldof=fb;ef=2



\end{lstlisting}



\mjmtol{
}


\section{Conclusions}


\section{Supplemental Information}

\subsection{Computer Code}


\begin{lstlisting}


\end{lstlisting}
\section{Bibliography}


\bibliography{\mjmbasename,\mjmaddbio}
\bibliographystyle{plainurl}


%%%%%%%%%%%%%%%%%%%%%%%%%%%%%%%%%%%%%%%%%%%%%%%%%%%%%%%%%%%%%%%%%%%%%%%%%%%%%
\begin{acknowledgments} 

% \input{generalack.tex}
\begin{enumerate}
\item Pubmed eutils facilities and the basic research it provides. 
\item Free software including Linux, R, LaTex  etc.
\item Thanks everyone who contributed incidental support. 
\end{enumerate}

\end{acknowledgments}

%%%%%%%%%%%%%%%%%%%%%%%%%%%%%%%%%%%%%%%%%%%%%%%%%%%%%%%%%%%%%%%%%%%%%%%%%%%%%
\clearpage
\appendix

\begin{mdpicomment}

\section{ Statement of Conflicts }
 No specific funding was used in this effort and there are no relationships
with others that could create a conflict of interest. I would like to develop
these ideas further and have obvious bias towards making them appear 
successful. Barbara Cade, the dog owner, has worked in the pet food industry
but this does not likely create a conflict. We have no interest in the
makers of any of the products named in this work.  

\end{mdpicomment}

\begin{mdpicomment}
\section{About the Authors and Facility}
This work was performed at a dog rescue run by Barbara Cade and
housed in rural Georgia.  The author of this report 
,Mike Marchywka,
has a background in electrical engineering and 
has done extensive research using free online literature sources.  
I hope to find additional people interested in critically 
examining the results and verify that they can be reproduced
effectively to treat other dogs.

\begin{comment}
\begin{figure}[htb] 
\centering
\mjmed{ picture commented out to save space in drafts...  } 
%\includegraphics[width=\picwidth]{me_on_brick.jpg}
\caption{ 
 }
\end{figure}

\end{comment}


\section{Symbols, Abbreviations and Colloquialisms}

\begin{comment}
% grep "[A-Z][A-Z]" paradox.tex | sed -e 's/[^A-Z]/\n/g' | grep "[A-Z]" | sort | uniq -c
% cat  paradox.tex | sed -e 's/  */\n/g' | grep "[A-Z][A-Z]"  | grep -v "[^A-Z]" | sort | uniq  |awk '{print $0" &   \\\\"; }'
\end{comment}


%\abbreviations{The following abbreviations are used in this manuscript:\\
%\begin{table}
\noindent
\begin{tabular}{@{}ll}
%SMVT & Sodium dependent Multi-Vitamin Transporter\\
TERM & definition and meaning   \\
\hline
%TLA & Three letter acronym\\
%LD & linear dichroism
\end{tabular} % }
%\end{table}

% https://tex.stackexchange.com/questions/5957/bibtex-entry-for-white-papers-and-technical-reports

\section{General caveats and disclaimer }
\label{appendix:caveats}

%\input{disclaimer-informal.tex}
\input{disclaimer-gaslight.tex}
\input{disclaimer-status.tex}

\section{Citing this as a tech report or white paper }
\label{appendix:citing}

Note: This is mostly manually entered and not assured to be error free.

This is tech report \mjmtrno. 

\begin{table}[H] \centering
\begin{tabular}{r|r|c|r}
Version & Date & Comments  &  \\
0.01 & \mjmmakedate  &  Create from empty.tex template  &  \\
-  & \today & version  \mjmversion { }   \mjmtrno  &  \\
1.0 & 20xx-xx-xx & First revision for distribution &  \\
\end{tabular}
\end{table}


Released versions,

build script needs to include empty releases.tex
\begin{table}[H] \centering
\begin{tabular}{|r|r|l|}
Version & Date & URL    \\
\hline
&  &  \\
% version & date & url  \\
%.1 table & 2021-08-17& {\url{https://www.linkedin.com/posts/marchywka_draft-compare-72020-theory-with-interim-activity-6833343119203860480--wJv}} \\
%.1 table & 2021-08-17& {\url{https://www.researchgate.net/publication/353946686_Draft_table_comparing_expectations_to_recent_results_with_covid-19}} \\
%.1 table & 2021-08-17 & {\url{https://www.academia.edu/s/34e160cae9}} \\

\hline
\end{tabular}
\end{table}





% 2020-11-30 keep on same page 
%\input{bibtex2.txt}

\begin{minipage}{\linewidth}
%\input{bibtex2.txt}
%\input{bibtex3.txt}
\mjmshowbib
\end{minipage}




\begin{comment}

{@}TECHREPORT\{mmarchywka-\mjmtrno,

  AUTHOR =        \{M.J. Marchywka\},

  TITLE =         \{ \mjmtitle \},
 
  NUMBER =        \{\mjmtrno\},
  
  VERSION =        \{\mjmversion\},

  INSTITUTION =   \{  not institutionalized , independent\},

  ADDRESS =       \{306 Charles Cox , Canton GA 30115\},

  NOTE =          \{ May be recycled in appropriate media \},

  DATE  =          \{\today\},

  DAY  =          \{\mjmbibday\},

  MONTH =         \{\mjmbibmo\},

  YEAR  =         \{\mjmbibyear\},

 % AUTHOR1EMAIL = \{marchywka@hotmail.com\},
  AUTHOR1EMAIL = \{\mjmemail\},

  AUTHOR1ID = \{orcid.org/0000-0001-9237-455X\},

%   PAGES =         \{6\},
  PAGES =         \{ \input{\mjmbasename.last_page} \},

  CONTACT =       \{marchywka@hotmail.com\},

  FILENAME =       \{\mjmbasename\}

\}
\end{comment}
\vspace{1cm}
Supporting files. Note that some dates,sizes, and md5's will change as this is
rebuilt.

This really needs to include the data analysis code 
but right now it is auto generated picking up things from prior
build in many cases 
\lstinputlisting{\mjmbasename.bundle_checksums}
\end{mdpicomment}
\end{document}
